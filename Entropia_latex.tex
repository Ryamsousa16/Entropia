\documentclass{article}
\usepackage{amsmath}
\usepackage{underscore}

\begin{document}

\title{Trabalho de entropia}
\author{Ryam Sousa e João Lucas}
\date{26/05/2023}

\maketitle

\section{Introdução}
A entropia é uma medida de incerteza em um conjunto de eventos probabilísticos. Ela é amplamente utilizada em teoria da informação e estatística para quantificar o grau de imprevisibilidade de um sistema. A fórmula geral da entropia é dada por:

\[
H = -\sum p(x) . (\log2^\text{(p(x))})
\]
\[\newline\]
Onde $H$ representa a entropia e $p$ representa as probabilidades dos eventos.

\section{Exemplo}
Considere um sistema com duas possibilidades: sim e não. Vamos calcular a entropia desse sistema usando a fórmula da entropia.

\subsection{Cálculo das probabilidades}
Suponha que a primeira possibilidade ("sim") ocorreu em 328 casos, enquanto a segunda possibilidade ("não") ocorreu em 396 casos.
\[\newline\]
Calculamos as probabilidades de cada possibilidade dividindo o número de casos de cada possibilidade pelo número total de casos:

\[
p_{\text{sim}} = \frac{328}{328 + 396} = \frac{328}{724} = 0.453
\]

\[
p_{\text{não}} = \frac{396}{328 + 396} = \frac{396}{724} = 0.546
\]

\subsection{Cálculo da entropia}
Substituindo as probabilidades na fórmula da entropia, temos:

\[
S = - (0.453 \log2(0.453) + 0.546 \log2(0.546))
\]
\[\newline\]
Agora, podemos calcular numericamente o valor da entropia.

\subsection{Resultados}
Realizando os cálculos, encontramos que a entropia do sistema é igual a $S \approx 0.994$.
\[\newline\]
Realizando os cálculos do máximo valor que a entropia pode alcançar, encontramos o resultado $S = 1$.



\section{Conclusão}
A entropia é uma medida útil para avaliar a imprevisibilidade de um sistema. Quanto maior a entropia, maior é o grau de incerteza e imprevisibilidade. Neste exemplo, a imprevisibilidade é alta, quase no limite possível, sendo um ótimo exemplo para aplicação de inteligência artificial.
\newpage
\section{Código em python}
import pandas as pd\\
from math import log
\[\newline\]
contadorN = 0 \\
contadorY = 0 \\

\hspace{-0.5cm}df = pd.read\_csv('heart.csv') 

\hspace{-0.5cm}df = df.drop(columns=['Cholesterol', 'ChestPainType', 'Oldpeak', 'ST\_Slope', 'RestingBP', 'MaxHR', 'HeartDisease',
'RestingECG', 'FastingBS', \par'HeartDisease', 'Age'])\\\\
df = df.dropna()\\

\hspace{-0.5cm}for i in range(0, 917):\par
    \hspace{0.5cm}sexo = df["Sex"][i]\[\newline\]

    \hspace{0.5cm}if sexo == "F":\par
        \hspace{1.5cm}df = df.drop(i)\par
    \hspace{0.5cm}else:\par
        \hspace{1.5cm}escolha = df["ExerciseAngina"][i]\par
        \hspace{1.5cm}if escolha == "N":\par
            \hspace{2.5cm}contadorN += 1 \par
        \hspace{1.5cm}else:\par
            \hspace{2.5cm}contadorY += 1\\ \par

dados = df.value\_counts() \par

list\_class = [contadorN, contadorY] \par
list\_porc = []\\ \par

quant\_result = contadorY + contadorN \par
quant\_class = 2\par
entropia = 0 \\ \par

for i in range(0, quant\_class):\par
    \hspace{0.5cm}x = list\_class[i] / quant\_result \par
    \hspace{0.5cm}list\_porc.append(x)\\ \par

for i in range(0, quant\_class):\par
    \hspace{0.5cm}porc = list\_porc[i]\par
    \hspace{0.5cm}x = log(porc) / log(2) \par
    \hspace{0.5cm}entropia += porc * x \par

print("O resultado da entropia é: {:.4}".format(entropia * -1)) \par
print("O valor máximo da entropia é: {:.4}".format(log(quant\_class, 2)))
\end{document}
